% 
% Annual Cognitive Science Conference
% Sample LaTeX Paper -- Proceedings Format
% 

% Original : Ashwin Ram (ashwin@cc.gatech.edu)       04/01/1994
% Modified : Johanna Moore (jmoore@cs.pitt.edu)      03/17/1995
% Modified : David Noelle (noelle@ucsd.edu)          03/15/1996
% Modified : Pat Langley (langley@cs.stanford.edu)   01/26/1997
% Latex2e corrections by Ramin Charles Nakisa        01/28/1997 
% Modified : Tina Eliassi-Rad (eliassi@cs.wisc.edu)  01/31/1998
% Modified : Trisha Yannuzzi (trisha@ircs.upenn.edu) 12/28/1999 (in process)
% Modified : Mary Ellen Foster (M.E.Foster@ed.ac.uk) 12/11/2000
% Modified : Ken Forbus                              01/23/2004
% Modified : Eli M. Silk (esilk@pitt.edu)            05/24/2005
% Modified : Niels Taatgen (taatgen@cmu.edu)         10/24/2006
% Modified : David Noelle (dnoelle@ucmerced.edu)     11/19/2014
% Modified : Roger Levy (rplevy@mit.edu)             12/31/2018



%% Change "letterpaper" in the following line to "a4paper" if you must.

\documentclass[10pt,letterpaper]{article}

\usepackage{cogsci}

%\cogscifinalcopy % Uncomment this line for the final submission 


\usepackage{pslatex}
\usepackage{apacite}
\usepackage{graphicx}
\usepackage[colorinlistoftodos]{todonotes}
\usepackage{float} % Roger Levy added this and changed figure/table
                   % placement to [H] for conformity to Word template,
                   % though floating tables and figures to top is
                   % still generally recommended!

%\usepackage[none]{hyphenat} % Sometimes it can be useful to turn off
%hyphenation for purposes such as spell checking of the resulting
%PDF.  Uncomment this block to turn off hyphenation.

\definecolor{Orange}{RGB}{255,140,0}
\newcommand{\ek}[1]{\textcolor{Orange}{[ek: #1]}} 


%\setlength\titlebox{4.5cm}
% You can expand the titlebox if you need extra space
% to show all the authors. Please do not make the titlebox
% smaller than 4.5cm (the original size).
%%If you do, we reserve the right to require you to change it back in
%%the camera-ready version, which could interfere with the timely
%%appearance of your paper in the Proceedings.



\title{Production Expectations Modulate Contrastive Inference}
 
\author{{\large \bf Elisa Kreiss (ekreiss@stanford.edu)} \\
  Department of Linguistics, 1202 W. Johnson Street \\
  Madison, WI 53706 USA
  \AND {\large \bf Judith Degen (jdegen@stanford.edu)} \\
  Department of Linguistics, 1025 W. Johnson Street \\
  Madison, WI 53706 USA}


\begin{document}

\maketitle

\begin{abstract}
Include no author information in the initial submission, to facilitate
blind review.  The abstract should be one paragraph, indented 1/8~inch on both sides,
in 9~point font with single spacing. The heading ``{\bf Abstract}''
should be 10~point, bold, centered, with one line of space below
it. This one-paragraph abstract section is required only for standard
six page proceedings papers. Following the abstract should be a blank
line, followed by the header ``{\bf Keywords:}'' and a list of
descriptive keywords separated by semicolons, all in 9~point font, as
shown below.

\textbf{Keywords:} 
add your choice of indexing terms or keywords; kindly use a
semicolon; between each term
\end{abstract}

\section{Introduction}

One of the most interesting features of language is its flexibility. To refer to one single object a speaker can choose an utterance out of an indefinite set of possible referring expressions. \textit{The banana}, \textit{the yellow banana}, \textit{the yellow, curvy fruit-thingy} for example are all possible utterances to refer to the intended target object in \ek{figure 1}. But this flexibility, which gives a speaker's language so much power, provides a challenge for the listener. In comprehension all of these utterances still need to be interpreted as describing this same object in this particular context in \ek{1}. Consequently, trying to understand how a listener processes these utterances and what affects their interpretation has become a central question in psycholinguistic research.

One of the most fundamental findings is that listeners process utterances incrementally, i.e., new information is incorporated into the interpretation of the utterance as soon as it becomes available \ek{cite}. For instance, eye-tracking experiments have shown that if a listener hears the incomplete utterance \textit{the yellow}, they will already focus on the yellow objects in the display before they hear the disambiguating noun \textit{banana} and arrive at the final referent \ek{cite}. 

But listeners also go beyond the information contained in the signal itself. In addition, they nourish the information provided by the context \ek{(as well as common ground...)} \ek{cite} and use these to rapidly draw pragmatic inferences about the speaker’s intention \ek{cite}. For instance if a listener was provided with the context in \ek{Fig 1} and heard the incomplete expression \textit{the red}, they would not only focus on all red objects, but also draw the pragmatic inference that the noun is most likely \textit{strawberry}, since there is no other red object in the context \ek{cite}.

If we want to understand comprehension, we need to understand how listeners can draw these rapid pragmatic inferences and which information listeners recruit to do so. One experimental design that is used to investigate this question is the ``contrast inference paradigm''. It was introduced by \ek{Sedivy 1999} as an eye-tracking experiment using the visual world paradigm. It introduces the idea that if a speaker chooses to use a modifier such as \textit{tall} in the utterance \textit{the tall glass}, they do so to contrast it with a smaller glass in the context. By presenting another tall object, such as a pitcher, in the same context, \ek{Sedivy} showed that before hearing the noun, listeners already showed preference for the glass over the pitcher. If we think the speaker is being cooperative \ek{cite Grice}, we expect them to not include more information than necessary in their referring expression — they could have referred to the pitcher more simply as just “the pitcher”. The adjective is only necessary for the purpose of distinguishing the tall from the short glass. These results suggest that the listeners drew a pragmatic inference, since the presence of the smaller glass led to a preference for the taller glass over the pitcher.

Especially in the size adjective domain, the contrastive inference effect has been replicated reliably \ek{as in... cite}.

But what could give rise to this inference? 
% CI is shown to be modulated by multiple factors, including adjective semantics [2], property salience [3], speaker reliability [4,5], and expectations of informativity [6] \ek{cite}.

Following recent research highlighting the importance of the listener’s generative model of the speaker in generating pragmatic inferences \ek{cite}, we propose the Rational-Speech Act (RSA) framework \ek{cite} as a way to think about contrastive inference incrementally. In this framework, the listener reasons about a speaker's possible utterances, therefore giving the speaker model a central role in the predictions. It provides a way to quantitatively assess which predictions a listener with prior beliefs and expectations about the speaker \textit{should} draw. This shifts the focus away from factors that influence contrastive inference production onto listener's production expectations (and their prior beliefs).

% \ek{Put this in Related Work} Previous work that addressed the question on what gives rise to a contrastive inference indirectly put more or less relevance onto the speaker. \ek{Sedivy 1999} for instance only considers a very limited speaker model. In this account, a contrastive inference arises when the modifier is not a component of the object's \textit{default description}. In other words, since \textit{tall} is added to the default description \textit{the glass}, the adjective elicits a contrastive inference in the listener. It therefore only assumes speaker considerations as to the creation of the default descriptions for objects and is completely independent of the context the target is presented in. \ek{explain color case here}

% \ek{Put this in Related Work} \ek{cite rubio fernandez} gives the speaker a more central role. In this account, the production probabilities of this more involved speaker directly affect the size of the contrastive inference. It therefore predicts that contrastive inference is not binary, but can occur in different strengths. While this account considers listeners' reasoning not only about the target but also about the competitor, the speaker model is still restricted to target considerations only. \ek{write this better}

In this paper we provide evidence for a strong pragmatic account of online processing, where the speaker model takes the central role. According to this account, we formulate the following \textbf{expectation-based hypothesis}: the strength of contrastive inferences triggered by an adjective can be predicted by the relative probability that the speaker would produce the observed adjective to convey the target \textbf{relative to the competitor}. When we put the listener's generative model of the speaker more into focus, we therefore predict that the production probabilities not just for the target, but also for the competitor play an important role. We use an incremental RSA model to make predictions about the size of the contrastive inference in the color domain by varying the empirical production expectations for the target and the competitor. We test these predictions by replicating the contrastive inference paradigm in an incremental decision task \ek{cite}.


\section{Related Work}

Since the introduction of the paradigm in \ek{Sedivy 1999}, the contrastive inference effect has been replicated especially in the size adjective domain \ek{cite} and has been shown to be modulated by multiple factors. \ek{CI is shown to be modulated by multiple factors, including adjective semantics [2], property salience [3], speaker reliability [4,5], and expectations of informativity [6] \ek{cite}.} 

These accounts which investigate specific factors that give rise to a contrastive inference indirectly put more or less relevance onto the speaker.
\ek{Sedivy 1999} for instance only considers a very limited speaker model. In this account, a contrastive inference arises when the modifier is not a component of the object's \textit{default description}. In other words, since \textit{tall} is added to the default description \textit{the glass}, the adjective elicits a contrastive inference in the listener. It therefore only assumes speaker considerations as to the creation of the default descriptions for objects and is completely independent of the context the target is presented in. \ek{explain color case here}

\ek{cite rubio fernandez} gives the speaker a more central role. In this account, the production probabilities of this more involved speaker directly affect the size of the contrastive inference. It therefore predicts that contrastive inference is not binary, but can occur in different strengths. While this account considers listeners' reasoning not only about the target but also about the competitor, the speaker model is still restricted to target considerations only. \ek{write this better}

In this work, we use the RSA framework to quantitatively assess how a highly pragmatic listener with a generative speaker model should draw contrastive inferences. \ek{cite 7, 8} \ek{explain general idea of RSA}


\section{Model}

\ek{RSA model here, formula; how we define the prior,...
how is it incremental? NEW
example predictions;
talk about Westerbeek study}

To investigate when a listener with a generative speaker model should draw a contrastive inference, we need to elicit how likely a listener can expect a modified over an unmodified referring expression for each object in the display. To evaluate the performance of the model and to gain information about the prior of the objects, we need comprehension data that informs us which object is considered the most likely target referent.

\section{Production Experiment: An Interactive Reference Game}

To investigate whether production probabilities affect listeners' contrastive inferences, we need to manipulate how likely a listener is to observe a modified utterance for each object in the display. Our main assumption is that the typicality of a color for an object will affect these modifier production probabilities. When the object is in isolation for instance, a listener should expect the speaker to call a yellow banana simply \textit{the banana}, but a blue banana \textit{the blue banana}. In this production experiment, we tested this hypothesis and empirically elicited the proportion of color modifier mention for the target and competitor. The results are taken as the basis of what production probabilities a listener can expect to observe.


\subsection{Participants}
We recruited 282 participants over Amazon Mechanical Turk, who were randomly matched to form one listener-speaker chat pair (i.e., 141 pairs in total). 
The estimated time for completion was 10 to 12 minutes and each participant was paid \$2.30. We restricted participation to workers with IP addresses in the US and an approval rate of previous work above 97\%.

Exclusions were performed on the 141 speakers, since they provided the utterances. Participants were excluded when they participated multiple times in the experiment (1 participant; 139 pairs remaining) and when they did not use a noun from the display in at least half of the cases (27 participants; 112 pairs remaining). These participants clearly misunderstood the task, using expressions such as \textit{yellow monkey} instead of \textit{yellow banana}, or \textit{should be yellow, must have teeth to eat} for \textit{corn}. All speakers indicated that their native language was English.


\subsection{Material}
Each context included four items, as displayed in \ek{Figure}. The pool of items consisted of 10 types (banana, broccoli, carrot, corn, egg, lettuce, pumpkin, strawberry, swan, tomato), each of which could occur in a typical or atypical color. For example, the broccoli could occur in its typical color green or in the atypical color red. The resulting pool contains 20 items, 10 of which are atypically colored. The number of colors is carefully counterbalanced such as each color occurs twice as a typical and twice as an atypical instance. All items were carefully normed for color-diagnosticity \ek{Tanaka Presnell}, typicality and nameability.


\subsection{Design}
The contexts varied in the typicality of the target, the typicality of the competitor and the presence of a contrast, resulting in eight conditions. For the critical trials, each participant saw four randomly created contexts from each of these eight conditions. In each condition we are interested in the modifier production probabilities for the target and the competitor, which is why in half of the trials when a contrast was present, the target was marked as the item to be communicated and in the other half, the competitor was marked. However when there is no contrast present, this distinction is irrelevant. For example, when target and competitor are both (a)typical, it is irrelevant, which is underlyingly coded as the target. Similarly when the modifier production probability for a typical target in context with an atypical competitor is the same as the probability for a typical competitor in a context with an atypical target. 
The fillers were eight randomly created contexts where the contrast was the item to be communicated and 20 randomly created contexts where the distractor was the item to be communicated.
Overall, each participant saw 60 different contexts (32 critical trials) in a completely randomized order.


\subsection{Procedure}
Participants were randomly paired up and each was randomly assigned either to the role of a speaker or listener. They could communicate freely through a real-time multi-player interface similar to \ek{Hawkins (2015)}. The speaker was instructed to communicate a target object out of a four-object context to the listener. The target could be identified by a green border surrounding it. The speaker and the listener saw the same set of objects but in a randomized order to avoid trivial position-based references such as “the left one”. After the listener clicked on the presumed target, both the speaker and listener received feedback about whether the right object had been selected.

\begin{figure}[H]
	\begin{center}
		\includegraphics[width=.475\textwidth]{graphs/prod-design.pdf}
	\end{center}
\caption{This is a figure. \ek{make text bigger in figure}} 
\label{prod-results}
\end{figure}


\subsection{Results}
\ek{only selected items with correct selection}
Figure~\ref{prod-results} shows the probability of color modifier mention for the target and competitor in each condition\footnote{Note that some data is duplicated in the conditions where the contrast is absent (\ek{see section...for explanation}). In the conditions where target and competitor are both (a)typical, the modifier probabilities are created by the same data. The underlying data is also identical in the two conditions where one of the items is (a)typical.}. 

\begin{figure}[H]
	\begin{center}
		\includegraphics[width=.45\textwidth]{graphs/prod-bycond-paper.pdf}
	\end{center}
\caption{This is a figure. \ek{make text bigger in figure}} 
\label{prod-results}
\end{figure}

When a contrast to the target is present (e.g., another banana), a speaker needs to include the color modifier to fully disambiguate the two items (see the upper row in Figure~\ref{prod-results}). When the typicality of the target is atypical, this is completely borne out in the data. However if the target is typical, participants sometimes also used the unmodified utterance. \ek{refer to speculations about the reason for this in discussion} 

When the contrast is absent (see the lower row in Figure~\ref{prod-results}), speakers were more likely to include a color modifier when referring to an atypical target than a typical one \ek{stats?}.

Independent of contrast, speakers were more likely to include the color modifier for an atypical color competitor over a typical one \ek{stats?}.

The results of this production experiment show that the probability of a speaker's modifier use is modulated by the color typicality of the item and the presence of a contrast. Our experiment therefore manipulates the modifier production probabilities a listener can expect in different contexts.


\section{Comprehension Experiment: An Incremental Decision Task}
To investigate which objects listeners consider to be the most likely target after observing the color adjective, we conducted an incremental decision task. 


\subsection{Participants}
We recruited 239 participants over Amazon's Mechanical Turk, 121 of which saw atypical color competitors and 118 saw typical color competitors in the critical trials. The study took on average 7 minutes and each of them were paid \$1.80 for their participation. We restricted participation to workers with IP addresses in the US and an approval rate of previous work above 97\%.

We excluded participants who did the Hit multiple times (1) who indicated that they did the Hit incorrectly or were confused (13), who indicated that they had a native language other than English (6) and who gave more then 20\% erroneous responses (7). An erroneous response is defined as a click to a non-target object after observing the fully disambiguating noun, i.e., participants are excluded who selected the wrong final object more than 11 times. Overall, we excluded 27 people, which is 11\% of the subjects. 211 participants remain, 108 of which were in the atypical competitor and 103 were in the typical competitor condition. 


\subsection{Material}
The item pool is the same as described in the production study \ek{Section..}.


\subsection{Design}
\ek{add choice of contexts (i.e., one per color,...); clarify what is within and between-subject manipulation with rationale}

\begin{figure}[H]
	\begin{center}
		\includegraphics[width=.475\textwidth]{graphs/IDT-design.pdf}
	\end{center}
\caption{This is a figure.} 
\label{prod-results}
\end{figure}


\subsection{Procedure}
This experiment is a one-player adaptation of the production study explained in \ek{ref to section} and follows the design of an incremental decision task \ek{cite Qing}. participants were put into the listener role of the reference game. That means, they needed to identify which object was the target given a referring expression placed above the grid. Crucially, they do not observe the complete referring expression at once, but instead the utterance is gradually revealed. After new information is revealed, participants are required to make their best guess onto which object is the most likely target. The choices had to be made prior any disambiguating information (after observing "Click on the"), after observing an adjective ("Click on the yellow") and after observing the fully disambiguating noun ("Click on the yellow banana!"). The clicks before information are observed are useful to determine whether there are already strong priors in item selections, even before any information were observed. The clicks after revealing the adjective are the critical clicks that will affect our interpretation of inferences drawn from the adjective. After observing the noun there is only one possible referent left. These clicks are used as attention checks.

Participants completed 55 trials in total, 20 of which were critical trials and 35 were fillers. The filler trials were supposed to ensure that participants perceive the referring expressions as being generated by a natural speaker. Firstly, all target trials are color modified utterances. To avoid that participants learn that the color modifier is always part of the referring expression, we need utterances that only have the bare noun. Second of all, we need to make sure that targets are not logically derivable. In a context such as \ek{Fig ref!}, the target can be determined by reasoning about the distractor choice. The target is the only object that shares the type (i.e., banana) with a distractor and its color (i.e., yellow) with another distractor. One can then derive that the object that is being distracted from will be the target. This regularity could also be learnt over the time course of the experiment. The filler trials therefore need to introduce primarily unmodified referring expressions that target other objects in the display. The exact trial structure is summarized in table \ref{tab:trialstructure}.

\begin{table}[]
	\begin{tabular}{llll}
	\textbf{trial type} & \textbf{number} & \textbf{utterance} & \textbf{referent}           \\
	critical            & 20              & modified           & target                      \\
	filler              & 5               & unmodified         & competitor (typical)  \\
	filler              & 5               & modified           & competitor (atypical) \\
	filler              & 5               & modified           & contrast                    \\
	filler              & 20              & unmodified         & distractor (typical)       
	\end{tabular}
	\vspace{2mm}
	\caption{Overview of the trial structure for the comprehension study.}
	\label{tab:trialstructure}
\end{table}

Before participants proceeded to the main trials, they had to complete four practice trials constructed from the speaker perspective. In the speaker role, they saw a grid of four non-color diagnostic objects, one of which was marked as the target by a green border surrounding it. They were then asked to refer to the object such that a second player could identify it. The practice trials were introduced to familiarize the participants with the task.

The main trials were randomized with one restriction: Trials in which we expect a color modifier to be superfluous or even misleading only occurred after the 15th trial. These were contexts where there was no contrast and either both target and color competitor typical objects, or the target typical while the color competitor was atypical. This measure should minimize the risk that participants perceive the "utterance generator" as unnatural.

% \subsection{Data Preprocessing and exclusion}

\subsection{Results}

\section{Discussion}

% There is not just a contrastive function on each adjective, but when a listener hears an adjective, it can be for multiple reasons, only one of which is contrastive. other reasons might be typicality, scene variation,... If we consider this to be true and the listener considers all objects in the display as potential targets, a lot of things need to be going alright for the target and competitor for the contrastive inference to happen. This also explains the difference between size and color adjectives, since size adjectives are not mentioned overinformatively.

% You can't take the results wrt presence/size of the contrastive inference effect and draw conclusions about the nature of adjective types. We can take color adjectives and make the contrastive inference (dis-)appear, i.e., we can make them look like size or material adjectives.

% We are showing that
% 1) listeners are highly pragmatic, i.e., they consider how well the utterance fits to each possible referent in the context and draws inferences about each and then takes them together -> competitor matters
% 2) thereby the modifier does not simply seem to receive a contrastive explanation for mentioning, but also a typicality explanation which affects the final inference just as much
% 3) taken together there simply seems to be an expectation of modifier inclusion which can result from multiple rationales.
% 4) we can make the contrastive inference strong (as for size adjectives) or disappear
% 5) in RSA we don't need to assume anything about the nature of the adjective, but can simply consider what the speaker probabilities for modifier use are... i.e., the linguistic knowledge can be obscure: we just need to consider production probabilities to predict listener behavior; underlyingly this could be simply communicative pressures; Due to informativity, the color needs to be mentioned if there is a contrast and the same holds for typicality (see PsychReview paper)






























% \section{General Formatting Instructions}

% The entire content of a paper (including figures, references, and anything else) can be no longer than six pages in the \textbf{initial submission}. In the \textbf{final submission}, the text of the paper, including an author line, must fit on six pages. Up to one additional page can be used for acknowledgements and references.

% The text of the paper should be formatted in two columns with an
% overall width of 7 inches (17.8 cm) and length of 9.25 inches (23.5
% cm), with 0.25 inches between the columns. Leave two line spaces
% between the last author listed and the text of the paper; the text of
% the paper (starting with the abstract) should begin no less than 2.75 inches below the top of the
% page. The left margin should be 0.75 inches and the top margin should
% be 1 inch.  \textbf{The right and bottom margins will depend on
%   whether you use U.S. letter or A4 paper, so you must be sure to
%   measure the width of the printed text.} Use 10~point Times Roman
% with 12~point vertical spacing, unless otherwise specified.

% The title should be in 14~point bold font, centered. The title should
% be formatted with initial caps (the first letter of content words
% capitalized and the rest lower case). In the initial submission, the
% phrase ``Anonymous CogSci submission'' should appear below the title,
% centered, in 11~point bold font.  In the final submission, each
% author's name should appear on a separate line, 11~point bold, and
% centered, with the author's email address in parentheses. Under each
% author's name list the author's affiliation and postal address in
% ordinary 10~point type.

% Indent the first line of each paragraph by 1/8~inch (except for the
% first paragraph of a new section). Do not add extra vertical space
% between paragraphs.


% \section{First Level Headings}

% First level headings should be in 12~point, initial caps, bold and
% centered. Leave one line space above the heading and 1/4~line space
% below the heading.


% \subsection{Second Level Headings}

% Second level headings should be 11~point, initial caps, bold, and
% flush left. Leave one line space above the heading and 1/4~line
% space below the heading.


% \subsubsection{Third Level Headings}

% Third level headings should be 10~point, initial caps, bold, and flush
% left. Leave one line space above the heading, but no space after the
% heading.


% \section{Formalities, Footnotes, and Floats}

% Use standard APA citation format. Citations within the text should
% include the author's last name and year. If the authors' names are
% included in the sentence, place only the year in parentheses, as in
% \citeA{NewellSimon1972a}, but otherwise place the entire reference in
% parentheses with the authors and year separated by a comma
% \cite{NewellSimon1972a}. List multiple references alphabetically and
% separate them by semicolons
% \cite{ChalnickBillman1988a,NewellSimon1972a}. Use the
% ``et~al.'' construction only after listing all the authors to a
% publication in an earlier reference and for citations with four or
% more authors.


% \subsection{Footnotes}

% Indicate footnotes with a number\footnote{Sample of the first
% footnote.} in the text. Place the footnotes in 9~point font at the
% bottom of the column on which they appear. Precede the footnote block
% with a horizontal rule.\footnote{Sample of the second footnote.}


% \subsection{Tables}

% Number tables consecutively. Place the table number and title (in
% 10~point) above the table with one line space above the caption and
% one line space below it, as in Table~\ref{sample-table}. You may float
% tables to the top or bottom of a column, and you may set wide tables across
% both columns.

% \begin{table}[H]
% \begin{center} 
% \caption{Sample table title.} 
% \label{sample-table} 
% \vskip 0.12in
% \begin{tabular}{ll} 
% \hline
% Error type    &  Example \\
% \hline
% Take smaller        &   63 - 44 = 21 \\
% Always borrow~~~~   &   96 - 42 = 34 \\
% 0 - N = N           &   70 - 47 = 37 \\
% 0 - N = 0           &   70 - 47 = 30 \\
% \hline
% \end{tabular} 
% \end{center} 
% \end{table}


% \subsection{Figures}

% All artwork must be very dark for purposes of reproduction and should
% not be hand drawn. Number figures sequentially, placing the figure
% number and caption, in 10~point, after the figure with one line space
% above the caption and one line space below it, as in
% Figure~\ref{sample-figure}. If necessary, leave extra white space at
% the bottom of the page to avoid splitting the figure and figure
% caption. You may float figures to the top or bottom of a column, and
% you may set wide figures across both columns.

% \begin{figure}[H]
% \begin{center}
% \fbox{CoGNiTiVe ScIeNcE}
% \end{center}
% \caption{This is a figure.} 
% \label{sample-figure}
% \end{figure}


% \section{Acknowledgments}

% In the \textbf{initial submission}, please \textbf{do not include
%   acknowledgements}, to preserve anonymity.  In the \textbf{final submission},
% place acknowledgments (including funding information) in a section \textbf{at
% the end of the paper}.


% \section{References Instructions}

% Follow the APA Publication Manual for citation format, both within the
% text and in the reference list, with the following exceptions: (a) do
% not cite the page numbers of any book, including chapters in edited
% volumes; (b) use the same format for unpublished references as for
% published ones. Alphabetize references by the surnames of the authors,
% with single author entries preceding multiple author entries. Order
% references by the same authors by the year of publication, with the
% earliest first.

% Use a first level section heading, ``{\bf References}'', as shown
% below. Use a hanging indent style, with the first line of the
% reference flush against the left margin and subsequent lines indented
% by 1/8~inch. Below are example references for a conference paper, book
% chapter, journal article, dissertation, book, technical report, and
% edited volume, respectively.

% \nocite{ChalnickBillman1988a}
% \nocite{Feigenbaum1963a}
% \nocite{Hill1983a}
% \nocite{OhlssonLangley1985a}
% % \nocite{Lewis1978a}
% \nocite{Matlock2001}
% \nocite{NewellSimon1972a}
% \nocite{ShragerLangley1990a}


\bibliographystyle{apacite}

\setlength{\bibleftmargin}{.125in}
\setlength{\bibindent}{-\bibleftmargin}

\bibliography{CogSci_Template}


\end{document}
